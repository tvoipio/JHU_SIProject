\documentclass[a4paper]{article}
\usepackage{lmodern}
\usepackage{amssymb,amsmath}
\usepackage{ifxetex,ifluatex}
\usepackage{fixltx2e} % provides \textsubscript
\ifnum 0\ifxetex 1\fi\ifluatex 1\fi=0 % if pdftex
  \usepackage[T1]{fontenc}
  \usepackage[utf8]{inputenc}
\else % if luatex or xelatex
  \ifxetex
    \usepackage{mathspec}
    \usepackage{xltxtra,xunicode}
  \else
    \usepackage{fontspec}
  \fi
  \defaultfontfeatures{Mapping=tex-text,Scale=MatchLowercase}
  \newcommand{\euro}{€}
    \setmainfont{texgyretermes-regular.otf}
    \setmathfont(Digits,Latin,Greek){texgyretermes-math.otf}
\fi
% use upquote if available, for straight quotes in verbatim environments
\IfFileExists{upquote.sty}{\usepackage{upquote}}{}
% use microtype if available
\IfFileExists{microtype.sty}{%
\usepackage{microtype}
\UseMicrotypeSet[protrusion]{basicmath} % disable protrusion for tt fonts
}{}
\usepackage[margin=1in]{geometry}
\ifxetex
  \usepackage[setpagesize=false, % page size defined by xetex
              unicode=false, % unicode breaks when used with xetex
              xetex]{hyperref}
\else
  \usepackage[unicode=true]{hyperref}
\fi
\hypersetup{breaklinks=true,
            bookmarks=true,
            pdfauthor={Timo Voipio},
            pdftitle={Effect of vitamin C on tooth cell growth},
            colorlinks=true,
            citecolor=blue,
            urlcolor=blue,
            linkcolor=magenta,
            pdfborder={0 0 0}}
\urlstyle{same}  % don't use monospace font for urls
\usepackage{color}
\usepackage{fancyvrb}
\newcommand{\VerbBar}{|}
\newcommand{\VERB}{\Verb[commandchars=\\\{\}]}
\DefineVerbatimEnvironment{Highlighting}{Verbatim}{commandchars=\\\{\}}
% Add ',fontsize=\small' for more characters per line
\usepackage{framed}
\definecolor{shadecolor}{RGB}{248,248,248}
\newenvironment{Shaded}{\begin{snugshade}}{\end{snugshade}}
\newcommand{\KeywordTok}[1]{\textcolor[rgb]{0.13,0.29,0.53}{\textbf{{#1}}}}
\newcommand{\DataTypeTok}[1]{\textcolor[rgb]{0.13,0.29,0.53}{{#1}}}
\newcommand{\DecValTok}[1]{\textcolor[rgb]{0.00,0.00,0.81}{{#1}}}
\newcommand{\BaseNTok}[1]{\textcolor[rgb]{0.00,0.00,0.81}{{#1}}}
\newcommand{\FloatTok}[1]{\textcolor[rgb]{0.00,0.00,0.81}{{#1}}}
\newcommand{\ConstantTok}[1]{\textcolor[rgb]{0.00,0.00,0.00}{{#1}}}
\newcommand{\CharTok}[1]{\textcolor[rgb]{0.31,0.60,0.02}{{#1}}}
\newcommand{\SpecialCharTok}[1]{\textcolor[rgb]{0.00,0.00,0.00}{{#1}}}
\newcommand{\StringTok}[1]{\textcolor[rgb]{0.31,0.60,0.02}{{#1}}}
\newcommand{\VerbatimStringTok}[1]{\textcolor[rgb]{0.31,0.60,0.02}{{#1}}}
\newcommand{\SpecialStringTok}[1]{\textcolor[rgb]{0.31,0.60,0.02}{{#1}}}
\newcommand{\ImportTok}[1]{{#1}}
\newcommand{\CommentTok}[1]{\textcolor[rgb]{0.56,0.35,0.01}{\textit{{#1}}}}
\newcommand{\DocumentationTok}[1]{\textcolor[rgb]{0.56,0.35,0.01}{\textbf{\textit{{#1}}}}}
\newcommand{\AnnotationTok}[1]{\textcolor[rgb]{0.56,0.35,0.01}{\textbf{\textit{{#1}}}}}
\newcommand{\CommentVarTok}[1]{\textcolor[rgb]{0.56,0.35,0.01}{\textbf{\textit{{#1}}}}}
\newcommand{\OtherTok}[1]{\textcolor[rgb]{0.56,0.35,0.01}{{#1}}}
\newcommand{\FunctionTok}[1]{\textcolor[rgb]{0.00,0.00,0.00}{{#1}}}
\newcommand{\VariableTok}[1]{\textcolor[rgb]{0.00,0.00,0.00}{{#1}}}
\newcommand{\ControlFlowTok}[1]{\textcolor[rgb]{0.13,0.29,0.53}{\textbf{{#1}}}}
\newcommand{\OperatorTok}[1]{\textcolor[rgb]{0.81,0.36,0.00}{\textbf{{#1}}}}
\newcommand{\BuiltInTok}[1]{{#1}}
\newcommand{\ExtensionTok}[1]{{#1}}
\newcommand{\PreprocessorTok}[1]{\textcolor[rgb]{0.56,0.35,0.01}{\textit{{#1}}}}
\newcommand{\AttributeTok}[1]{\textcolor[rgb]{0.77,0.63,0.00}{{#1}}}
\newcommand{\RegionMarkerTok}[1]{{#1}}
\newcommand{\InformationTok}[1]{\textcolor[rgb]{0.56,0.35,0.01}{\textbf{\textit{{#1}}}}}
\newcommand{\WarningTok}[1]{\textcolor[rgb]{0.56,0.35,0.01}{\textbf{\textit{{#1}}}}}
\newcommand{\AlertTok}[1]{\textcolor[rgb]{0.94,0.16,0.16}{{#1}}}
\newcommand{\ErrorTok}[1]{\textcolor[rgb]{0.64,0.00,0.00}{\textbf{{#1}}}}
\newcommand{\NormalTok}[1]{{#1}}
\usepackage{graphicx,grffile}
\makeatletter
\def\maxwidth{\ifdim\Gin@nat@width>\linewidth\linewidth\else\Gin@nat@width\fi}
\def\maxheight{\ifdim\Gin@nat@height>\textheight\textheight\else\Gin@nat@height\fi}
\makeatother
% Scale images if necessary, so that they will not overflow the page
% margins by default, and it is still possible to overwrite the defaults
% using explicit options in \includegraphics[width, height, ...]{}
\setkeys{Gin}{width=\maxwidth,height=\maxheight,keepaspectratio}
\setlength{\parindent}{0pt}
\setlength{\parskip}{6pt plus 2pt minus 1pt}
\setlength{\emergencystretch}{3em}  % prevent overfull lines
\providecommand{\tightlist}{%
  \setlength{\itemsep}{0pt}\setlength{\parskip}{0pt}}
\setcounter{secnumdepth}{0}

%%% Use protect on footnotes to avoid problems with footnotes in titles
\let\rmarkdownfootnote\footnote%
\def\footnote{\protect\rmarkdownfootnote}

%%% Change title format to be more compact
\usepackage{titling}

% Create subtitle command for use in maketitle
\newcommand{\subtitle}[1]{
  \posttitle{
    \begin{center}\large#1\end{center}
    }
}

\setlength{\droptitle}{-2em}
  \title{Effect of vitamin C on tooth cell growth}
  \pretitle{\vspace{\droptitle}\centering\huge}
  \posttitle{\par}
  \author{Timo Voipio}
  \preauthor{\centering\large\emph}
  \postauthor{\par}
  \predate{\centering\large\emph}
  \postdate{\par}
  \date{14 Aug 2016}

\usepackage{colortbl}
\usepackage{booktabs}

\setmainfont[
BoldFont = texgyretermes-bold.otf,
ItalicFont = texgyretermes-italic.otf,
BoldItalicFont = texgyretermes-bolditalic.otf
]{texgyretermes-regular.otf}

% Redefines (sub)paragraphs to behave more like sections
\ifx\paragraph\undefined\else
\let\oldparagraph\paragraph
\renewcommand{\paragraph}[1]{\oldparagraph{#1}\mbox{}}
\fi
\ifx\subparagraph\undefined\else
\let\oldsubparagraph\subparagraph
\renewcommand{\subparagraph}[1]{\oldsubparagraph{#1}\mbox{}}
\fi

\begin{document}
\maketitle

\section{Overview}\label{overview}

This paper explores the \texttt{ToothGrowth} dataset from the R
\texttt{datasets} package. Based on the data, we determine that larger
dosages of vitamin C is associated with longer odontoblasts, and that
for lower concentrations administering the vitamin via orange juice
seems to be associated with longer odontoblasts, compared to ascorbic
acid administration.

\section{Data description}\label{data-description}

The dataset is described in its documentation as follows: ``The Effect
of Vitamin C on Tooth Growth in Guinea Pigs. The response is the length
of odontoblasts (cells responsible for tooth growth) in 60 guinea pigs.
Each animal received one of three dose levels of vitamin C (0.5, 1.0,
and 2.0 mg/day) by one of two delivery methods, (orange juice or
ascorbic acid (a form of vitamin C and coded as VC).'' Administration
via orange juice is coded with ``OJ''. There are ten measurements of the
response (odontoblast length) for each of the six possible combinations
of dose and administration method.

\section{Exploratory analysis}\label{exploratory-analysis}

The measured odontoblast lengths are shown below as a function of the
vitamin C dose. Administration method is indicated by marker color. The
mean of each of the six dose--method groups is indicated with a cross,
colored according to the method.

\begin{center}\includegraphics{JHU_StatInf_project_tooth_files/figure-latex/exploreplot2-1} \end{center}

We observe that the odontoblasts seem to be longer in those gerbils
which had larger daily dose of vitamin C. For doses of 0.5 and 1.0
mg/day, also the administration method seems to have an effect, with the
orange juice administration being associated with longer odontoblasts.

\section{Hypothesis testing of the mean odontoblast
length}\label{hypothesis-testing-of-the-mean-odontoblast-length}

We investigate the differences in the mean lengths of different
dose--method groups by using the two-sided \emph{t}-test
(\(\alpha = 0.05\)). In order to control false discovery rate, the
p-values are adjusted using the Benjamini--Hochberg method, R function
\texttt{p.adjust(p,\ method\ =\ "BH")}. Only the dose--method pairs
where either the dose or the method are the same are considered, i.e.,
if both the dose and the administration method differ the difference of
the means and the associated p-value is not calculated.

The differences in the means are given in the table below. Gray cells
correspond to pairs which are not considered. Cells with green
background indicate that the difference in the means is statistically
significant, while red background indicates the converse. The difference
of the mean is given first, positive value indicates that the mean of
the group named on the same row is larger than the mean of the group
named at the top of the column. The number after the semicolon is the
(unadjusted) p-value.

\begin{table}[ht]
\centering
\begingroup\small
\begin{tabular}{rccccc}
  \toprule
 & VC 0.5 & OJ 0.5 & VC 1.0 & OJ 1.0 & VC 2.0 \\ 
  \midrule
OJ 0.5 & \cellcolor[RGB]{153,255,153}  $5.25$; $6.36 \times 10^{-3}$ & \cellcolor[gray]{0.9}  & \cellcolor[gray]{0.9}  & \cellcolor[gray]{0.9}  & \cellcolor[gray]{0.9}  \\ 
  VC 1.0 & \cellcolor[RGB]{153,255,153}  $8.79$; $6.81 \times 10^{-7}$ & \cellcolor[gray]{0.9}  & \cellcolor[gray]{0.9}  & \cellcolor[gray]{0.9}  & \cellcolor[gray]{0.9}  \\ 
  OJ 1.0 & \cellcolor[gray]{0.9}  & \cellcolor[RGB]{153,255,153}  $9.47$; $8.78 \times 10^{-5}$ & \cellcolor[RGB]{153,255,153}  $5.93$; $1.04 \times 10^{-3}$ & \cellcolor[gray]{0.9}  & \cellcolor[gray]{0.9}  \\ 
  VC 2.0 & \cellcolor[RGB]{153,255,153} $18.16$; $4.68 \times 10^{-8}$ & \cellcolor[gray]{0.9}  & \cellcolor[RGB]{153,255,153}  $9.37$; $9.16 \times 10^{-5}$ & \cellcolor[gray]{0.9}  & \cellcolor[gray]{0.9}  \\ 
  OJ 2.0 & \cellcolor[gray]{0.9}  & \cellcolor[RGB]{153,255,153} $12.83$; $1.32 \times 10^{-6}$ & \cellcolor[gray]{0.9}  & \cellcolor[RGB]{153,255,153}  $3.36$; $3.92 \times 10^{-2}$ & \cellcolor[RGB]{255,173,153} $-0.08$; $9.64 \times 10^{-1}$ \\ 
   \bottomrule
\end{tabular}
\endgroup
\end{table}

We observe that for both administration methods and all combinations of
dose, the difference in the means is statistically significant.
Additionally, for daily doses of 0.5 and 1.0 mg, the gerbils to whom the
vitamin was administered via orange juice had, on average, longer
odontoblast cells than those who were administered ascorbic acid. For
the daily dose of 2.0 mg, the difference between the administration
methods was not statistically significant.

\section{Permutation testing}\label{permutation-testing}

Finally, we evaluate the different administration methods via
permutation testing at each dose level. For each 3 dose levels, the
administration method labels are reassigned at random 10000 times, then
the difference between the means of the reassigned groups is calculated
and compared to the difference in the original data. The fraction of the
permutations where the difference of means is greater than the actual
sample mean serves as an estimate of the significance of the measured
difference.

The following table shows the result of the permutation testing. From
the previous section our hypothesis is that orange juice (OJ) is
associated with higher mean than ascorbic acid (VC), so for each
permutation we have subtracted the mean of the data labeled \texttt{VC}
from the mean of the data assigned the label \texttt{OJ}. The observed
difference for each dose is given in the column \(\hat{\Delta}\). The
mean and standard deviation of the differences obtained by permuting the
labels are given in the columns \(\bar{\Delta}\) and \(\sigma\),
respectively. The rightmost column \(\hat{\Delta} > \bar{\Delta}\) shows
the fraction of the permutations which resulted in larger difference of
means than present in the original data.

\begin{table}[ht]
\centering
\begin{tabular}{rrrrr}
  \toprule
Dose & $\hat{\Delta}$ & $\bar{\Delta}$ & $\sigma$ & $\bar{\Delta} > \hat{\Delta}$ \\ 
  \midrule
0.5 & 5.25 & 0.01 & 2.01 & 0.0031 \\ 
  1.0 & 5.93 & 0.01 & 1.97 & 0.0010 \\ 
  2.0 & -0.08 & 0.03 & 1.70 & 0.5247 \\ 
   \bottomrule
\end{tabular}
\end{table}

Small values of \(\hat{\Delta} > \bar{\Delta}\) indicate that the
difference in the means between the administration methods, as in the
original data, is statistically significant. We see that for the two
smaller doses the difference between the administration methods, is
significant, but the largest dose approximately half of the label
permutations result in a difference of means larger than in the original
data.

\section{Conclusions}\label{conclusions}

We conclude that, according to our analysis, larger doses of vitamin C
are associated with longer odontoblasts. For daily doses of 0.5 mg and
1.0 mg, administering the vitamin via orange juice instead of as
ascorbic acid is associated higher mean odontoblast length. These
results are statistically significant (\(\alpha < 0.05\)); the
Benjamini--Hochberg correction was used to control the false discovery
rate. For the largest reported dose, 2.0 mg/day, the difference between
the administration methods is not statistically significant. Our
conclusions are based on the assumption that the odontoblast length is
normally distributed and thus the \emph{t}-test may be used.

\section{Appendix}\label{appendix}

This appendix includes the R code used to conduct the simulation
experiment and to format the results. The simulation was performed using
R version 3.2.3 (2015-12-10) on OS X 10.10.5 (Yosemite)
{[}x86\_64-apple-darwin13.4.0 (64-bit){]}.

\subsection{Source code}\label{source-code}

The following shows the source code used in the analysis. Due to length
constraints, some code sections related strictly to presenting the
results have been omitted. Complete R Markdown sourcecode is available
in GitHub, \url{https://github.com/tvoipio/JHU_SIProject}

\begin{Shaded}
\begin{Highlighting}[]
\KeywordTok{library}\NormalTok{(ggplot2)}
\KeywordTok{library}\NormalTok{(datasets)}
\KeywordTok{library}\NormalTok{(xtable)}

\CommentTok{# Temporarily disable warnings while dplyr is loaded}
\NormalTok{oldw <-}\StringTok{ }\KeywordTok{getOption}\NormalTok{(}\StringTok{"warn"}\NormalTok{)}
\KeywordTok{options}\NormalTok{(}\DataTypeTok{warn =} \NormalTok{-}\DecValTok{1}\NormalTok{)}
\KeywordTok{library}\NormalTok{(dplyr)}
\KeywordTok{options}\NormalTok{(}\DataTypeTok{warn =} \NormalTok{oldw)}

\KeywordTok{data}\NormalTok{(}\StringTok{"ToothGrowth"}\NormalTok{)}
\CommentTok{#tg <- ToothGrowth[, c("len", "supp", "dose")]}

\CommentTok{# Set random seed for repeatability}
\KeywordTok{set.seed}\NormalTok{(}\DecValTok{160808}\NormalTok{)}
\CommentTok{# Determine the mean odontoblast length for each combination of dose}
\CommentTok{# and administration method}
\NormalTok{tgmeans <-}\StringTok{ }\KeywordTok{group_by}\NormalTok{(ToothGrowth, supp, dose) %>%}
\StringTok{    }\KeywordTok{summarize}\NormalTok{(}\DataTypeTok{len =} \KeywordTok{mean}\NormalTok{(len)) %>%}
\StringTok{    }\KeywordTok{ungroup}\NormalTok{()}
\CommentTok{# Ensure that the columns of tgmeans are in the same order as in tg}
\NormalTok{tgmeans <-}\StringTok{ }\NormalTok{tgmeans[, }\KeywordTok{names}\NormalTok{(ToothGrowth)]}

\CommentTok{# Combine the measurement results and the computed means for plotting}
\NormalTok{tgc <-}\StringTok{ }\KeywordTok{rbind}\NormalTok{(ToothGrowth, tgmeans)}

\CommentTok{# Add a factor variable to indicate whether the row is measured data}
\CommentTok{# or a mean}
\NormalTok{tgc <-}\StringTok{ }\KeywordTok{cbind}\NormalTok{(tgc,}
             \DataTypeTok{type =} \KeywordTok{factor}\NormalTok{(}\KeywordTok{c}\NormalTok{(}\KeywordTok{rep}\NormalTok{(}\StringTok{"Measured"}\NormalTok{, }\DataTypeTok{times =} \KeywordTok{nrow}\NormalTok{(ToothGrowth)),}
                             \KeywordTok{rep}\NormalTok{(}\StringTok{"Mean"}\NormalTok{, }\DataTypeTok{times =} \KeywordTok{nrow}\NormalTok{(tgmeans)))))}

\CommentTok{# Prepare a plot of length vs dose using ggplot2}
\NormalTok{gt <-}\StringTok{ }\KeywordTok{ggplot}\NormalTok{(tgc, }\KeywordTok{aes}\NormalTok{(}\DataTypeTok{x =} \NormalTok{dose, }\DataTypeTok{y =} \NormalTok{len))}
\end{Highlighting}
\end{Shaded}

(code block omitted)

\begin{Shaded}
\begin{Highlighting}[]
\CommentTok{# Create a data frame of the possible combinations of administration method}
\CommentTok{# and dose, one combination per row}
\CommentTok{# The leftmost variable variest fastest, the rightmost slowest}
\NormalTok{metxdose <-}\StringTok{ }\KeywordTok{expand.grid}\NormalTok{(}\DataTypeTok{supp =} \KeywordTok{unique}\NormalTok{(tgc$supp), }\DataTypeTok{dose =} \KeywordTok{unique}\NormalTok{(tgc$dose))}

\CommentTok{# Set the significance level}
\NormalTok{signif.level <-}\StringTok{ }\FloatTok{0.05}

\CommentTok{# Choose the p value adjustment method to control the false discovery}
\CommentTok{# rate (using p.adjust())}
\CommentTok{#adj.method <- "bonferroni"}
\NormalTok{adj.method <-}\StringTok{ "BH"}

\CommentTok{# Indices of the elements of a lower triangular matrix (main diagonal omitted)}
\CommentTok{# Thanks to: http://stackoverflow.com/a/20898910}
\CommentTok{# We want to consider each dose--method pair only once}
\NormalTok{rows <-}\StringTok{ }\KeywordTok{nrow}\NormalTok{(metxdose)}
\NormalTok{rowinds <-}\StringTok{ }\KeywordTok{rev}\NormalTok{(}\KeywordTok{abs}\NormalTok{(}\KeywordTok{sequence}\NormalTok{(}\KeywordTok{seq.int}\NormalTok{(rows -}\StringTok{ }\DecValTok{1}\NormalTok{)) -}\StringTok{ }\NormalTok{rows) +}\StringTok{ }\DecValTok{1}\NormalTok{)}
\NormalTok{colinds <-}\StringTok{ }\KeywordTok{rep.int}\NormalTok{(}\KeywordTok{seq.int}\NormalTok{(rows -}\StringTok{ }\DecValTok{1}\NormalTok{), }\KeywordTok{rev}\NormalTok{(}\KeywordTok{seq.int}\NormalTok{(rows -}\StringTok{ }\DecValTok{1}\NormalTok{)))}
\NormalTok{idx <-}\StringTok{ }\KeywordTok{cbind}\NormalTok{(rowinds, colinds)}

\CommentTok{# Form a list of the index pairs; each element of the list identifies}
\CommentTok{# a dose-method combination, i.e., a row in metxdose}
\NormalTok{idxl <-}\StringTok{ }\KeywordTok{lapply}\NormalTok{(}\KeywordTok{seq_len}\NormalTok{(}\KeywordTok{nrow}\NormalTok{(idx)), function(i) }\KeywordTok{as.vector}\NormalTok{(idx[i,]))}

\CommentTok{# Remove from consideration the indices which consider to pairs}
\CommentTok{# where both the dose and the administration method differ}
\NormalTok{onediffers <-}\StringTok{ }\KeywordTok{sapply}\NormalTok{(idxl, function(x) }\KeywordTok{any}\NormalTok{(metxdose[x[}\DecValTok{1}\NormalTok{], ]}
                                           \NormalTok{==}\StringTok{ }\NormalTok{metxdose[x[}\DecValTok{2}\NormalTok{], ]))}
\NormalTok{idxl <-}\StringTok{ }\NormalTok{idxl[onediffers]}

\CommentTok{# Function to calculate the two-sided t-test between the identified}
\CommentTok{# pair of dose-method combinations. The calculated difference is}
\CommentTok{# data.r-data.c, where data.c is the combination corresponding to the}
\CommentTok{# index on the column in the metxdose matrix and data.r corresponds to}
\CommentTok{# the row.}
\NormalTok{calculate_stat <-}\StringTok{ }\NormalTok{function(l)}
\NormalTok{\{}
    \NormalTok{data.r <-}\StringTok{ }\NormalTok{ToothGrowth[ToothGrowth$supp ==}\StringTok{ }\NormalTok{metxdose[l[}\DecValTok{1}\NormalTok{], }\StringTok{"supp"}\NormalTok{] &}
\StringTok{                  }\NormalTok{ToothGrowth$dose ==}\StringTok{ }\NormalTok{metxdose[l[}\DecValTok{1}\NormalTok{], }\StringTok{"dose"}\NormalTok{], }\StringTok{"len"}\NormalTok{]}
    \NormalTok{data.c <-}\StringTok{ }\NormalTok{ToothGrowth[ToothGrowth$supp ==}\StringTok{ }\NormalTok{metxdose[l[}\DecValTok{2}\NormalTok{], }\StringTok{"supp"}\NormalTok{] &}
\StringTok{                  }\NormalTok{ToothGrowth$dose ==}\StringTok{ }\NormalTok{metxdose[l[}\DecValTok{2}\NormalTok{], }\StringTok{"dose"}\NormalTok{], }\StringTok{"len"}\NormalTok{]}
    \KeywordTok{t.test}\NormalTok{(data.r, data.c, }\DataTypeTok{paired =} \OtherTok{FALSE}\NormalTok{, }\DataTypeTok{var.equal =} \OtherTok{FALSE}\NormalTok{)}
\NormalTok{\}}

\CommentTok{# Perform the t tests for each element in idxl, and for convenience}
\CommentTok{# separate the p values of the calculated differences of means}
\CommentTok{# and the differences themselves into separate vectors}
\NormalTok{t.test.results <-}\StringTok{ }\KeywordTok{lapply}\NormalTok{(idxl, calculate_stat)}
\NormalTok{p.values <-}\StringTok{ }\KeywordTok{sapply}\NormalTok{(t.test.results, function(res) res$p.value)}
\NormalTok{meandeltas <-}\StringTok{ }\KeywordTok{sapply}\NormalTok{(t.test.results, function(res) -}\KeywordTok{diff}\NormalTok{(res$estimate))}
\KeywordTok{names}\NormalTok{(meandeltas) <-}\StringTok{ }\OtherTok{NULL}

\CommentTok{# Calculate the adjusted p values}
\NormalTok{p.adj.values <-}\StringTok{ }\KeywordTok{p.adjust}\NormalTok{(p.values, }\DataTypeTok{method =} \NormalTok{adj.method)}

\CommentTok{# Create descriptive yet concise row and column names for use in the }
\CommentTok{# result matrix}
\NormalTok{rowcolnames <-}\StringTok{ }\KeywordTok{apply}\NormalTok{(metxdose, }\DecValTok{1}\NormalTok{, paste, }\DataTypeTok{collapse =} \StringTok{" "}\NormalTok{)}

\CommentTok{# Construct empty matrices into which the calculated p values, adjusted}
\CommentTok{# p values, mean deltas and whether the result is significant or not are placed}
\CommentTok{# Since the matrix data is given as a 0-length vector, they are filled}
\CommentTok{# with NAs.}
\NormalTok{p.values.mat <-}\StringTok{ }\KeywordTok{matrix}\NormalTok{(}\DataTypeTok{data =} \KeywordTok{numeric}\NormalTok{(), }\DataTypeTok{nrow =} \NormalTok{rows, }\DataTypeTok{ncol =} \NormalTok{rows,}
                       \DataTypeTok{dimnames =} \KeywordTok{list}\NormalTok{(rowcolnames, rowcolnames))}
\NormalTok{p.adj.values.mat <-}\StringTok{ }\NormalTok{p.values.mat}
\NormalTok{meandeltas.mat <-}\StringTok{ }\NormalTok{p.values.mat}
\NormalTok{p.signif.mat <-}\StringTok{ }\KeywordTok{matrix}\NormalTok{(}\DataTypeTok{data =} \KeywordTok{logical}\NormalTok{(), }\DataTypeTok{nrow =} \NormalTok{rows, }\DataTypeTok{ncol =} \NormalTok{rows,}
                       \DataTypeTok{dimnames =} \KeywordTok{list}\NormalTok{(rowcolnames, rowcolnames))}

\CommentTok{# Populate the matrices created above}
\NormalTok{for (i in }\KeywordTok{seq_along}\NormalTok{(idxl))}
\NormalTok{\{}
    \NormalTok{row <-}\StringTok{ }\NormalTok{idxl[[i]][}\DecValTok{1}\NormalTok{]}
    \NormalTok{col <-}\StringTok{ }\NormalTok{idxl[[i]][}\DecValTok{2}\NormalTok{]}
    \NormalTok{p.values.mat[row, col] <-}\StringTok{ }\NormalTok{p.values[i]}
    \NormalTok{p.adj.values.mat[row, col] <-}\StringTok{ }\NormalTok{p.adj.values[i]}
    \NormalTok{meandeltas.mat[row, col] <-}\StringTok{ }\NormalTok{meandeltas[i]}
    \NormalTok{p.signif.mat[row, col] <-}\StringTok{ }\NormalTok{p.adj.values[i] <}\StringTok{ }\NormalTok{signif.level}
\NormalTok{\}}
\KeywordTok{rm}\NormalTok{(row, col)}
\end{Highlighting}
\end{Shaded}

(code block omitted)

\begin{Shaded}
\begin{Highlighting}[]
\NormalTok{Nperm <-}\StringTok{ }\DecValTok{10000}
\NormalTok{doses <-}\StringTok{ }\KeywordTok{unique}\NormalTok{(ToothGrowth$dose)}
\CommentTok{# Create a data frame for storing the results of the permutation tests}
\NormalTok{perms <-}\StringTok{ }\KeywordTok{data.frame}\NormalTok{(}\KeywordTok{matrix}\NormalTok{(}\KeywordTok{numeric}\NormalTok{(Nperm*}\KeywordTok{length}\NormalTok{(doses)*}\DecValTok{2}\NormalTok{), }\DataTypeTok{ncol =} \DecValTok{2}\NormalTok{))}
\KeywordTok{names}\NormalTok{(perms) <-}\StringTok{ }\KeywordTok{c}\NormalTok{(}\StringTok{"deltamean"}\NormalTok{, }\StringTok{"dose"}\NormalTok{)}
\NormalTok{perms[, }\StringTok{"greater"}\NormalTok{] <-}\StringTok{ }\KeywordTok{logical}\NormalTok{()}

\NormalTok{testStat <-}\StringTok{ }\NormalTok{function(x, suppgrp)}
\NormalTok{\{}
    \KeywordTok{mean}\NormalTok{(x[suppgrp ==}\StringTok{ "OJ"}\NormalTok{, }\StringTok{"len"}\NormalTok{]) -}\StringTok{ }\KeywordTok{mean}\NormalTok{(x[suppgrp ==}\StringTok{ "VC"}\NormalTok{, }\StringTok{"len"}\NormalTok{])}
\NormalTok{\}}

\NormalTok{obsdeltas <-}\StringTok{ }\KeywordTok{numeric}\NormalTok{(}\DecValTok{3}\NormalTok{)}
    
\NormalTok{for (i in }\KeywordTok{seq_along}\NormalTok{(doses))}
\NormalTok{\{}
    \NormalTok{dosei <-}\StringTok{ }\NormalTok{doses[i]}
    \NormalTok{tgdose <-}\StringTok{ }\KeywordTok{subset}\NormalTok{(ToothGrowth, dose ==}\StringTok{ }\NormalTok{dosei)}

    \NormalTok{obsdeltas[i] <-}\StringTok{ }\KeywordTok{testStat}\NormalTok{(tgdose, tgdose$supp)}
    
    \NormalTok{offset <-}\StringTok{ }\NormalTok{(i}\DecValTok{-1}\NormalTok{)*Nperm}
    \NormalTok{perms[(offset +}\StringTok{ }\DecValTok{1}\NormalTok{:Nperm), }\StringTok{"deltamean"}\NormalTok{] <-}
\StringTok{        }\KeywordTok{sapply}\NormalTok{(}\KeywordTok{seq}\NormalTok{(Nperm), function(x) }\KeywordTok{testStat}\NormalTok{(tgdose, }\KeywordTok{sample}\NormalTok{(tgdose$supp)))}
    \NormalTok{perms[(offset +}\StringTok{ }\DecValTok{1}\NormalTok{:Nperm), }\StringTok{"dose"}\NormalTok{] <-}\StringTok{ }\NormalTok{dosei}
    \NormalTok{perms[(offset +}\StringTok{ }\DecValTok{1}\NormalTok{:Nperm), }\StringTok{"greater"}\NormalTok{] <-}
\StringTok{        }\NormalTok{perms[(offset +}\StringTok{ }\DecValTok{1}\NormalTok{:Nperm), }\StringTok{"deltamean"}\NormalTok{] >}\StringTok{ }\NormalTok{obsdeltas[i]}
\NormalTok{\}}

\NormalTok{dosetab <-}\StringTok{ }\NormalTok{perms %>%}
\StringTok{    }\KeywordTok{group_by}\NormalTok{(dose) %>%}
\StringTok{    }\KeywordTok{summarize}\NormalTok{(}\DataTypeTok{meandelta =} \KeywordTok{mean}\NormalTok{(deltamean), }\DataTypeTok{sddelta =} \KeywordTok{sd}\NormalTok{(deltamean),}
              \DataTypeTok{greater =} \KeywordTok{mean}\NormalTok{(greater))}
\NormalTok{dosetab <-}\StringTok{ }\KeywordTok{cbind}\NormalTok{(dosetab, }\DataTypeTok{obsdelta =} \NormalTok{obsdeltas)}
\NormalTok{dosetab <-}\StringTok{ }\KeywordTok{select}\NormalTok{(dosetab, dose, obsdelta, }\KeywordTok{everything}\NormalTok{())}
\KeywordTok{names}\NormalTok{(dosetab) <-}\StringTok{ }\KeywordTok{c}\NormalTok{(}\StringTok{"Dose"}\NormalTok{, }\StringTok{"$}\CharTok{\textbackslash{}\textbackslash{}}\StringTok{hat\{}\CharTok{\textbackslash{}\textbackslash{}}\StringTok{Delta\}$"}\NormalTok{, }\StringTok{"$}\CharTok{\textbackslash{}\textbackslash{}}\StringTok{bar\{}\CharTok{\textbackslash{}\textbackslash{}}\StringTok{Delta\}$"}\NormalTok{,}
                    \StringTok{"$}\CharTok{\textbackslash{}\textbackslash{}}\StringTok{sigma$"}\NormalTok{, }\StringTok{"$}\CharTok{\textbackslash{}\textbackslash{}}\StringTok{bar\{}\CharTok{\textbackslash{}\textbackslash{}}\StringTok{Delta\} > }\CharTok{\textbackslash{}\textbackslash{}}\StringTok{hat\{}\CharTok{\textbackslash{}\textbackslash{}}\StringTok{Delta\}$"}\NormalTok{)}
\KeywordTok{print}\NormalTok{(}\KeywordTok{xtable}\NormalTok{(dosetab, }\DataTypeTok{digits =} \KeywordTok{c}\NormalTok{(}\DecValTok{1}\NormalTok{, }\DecValTok{1}\NormalTok{, }\DecValTok{2}\NormalTok{, }\DecValTok{2}\NormalTok{, }\DecValTok{2}\NormalTok{, }\DecValTok{4}\NormalTok{)),}
      \DataTypeTok{include.rownames =} \OtherTok{FALSE}\NormalTok{, }\DataTypeTok{sanitize.colnames.function =} \NormalTok{identity)}
\end{Highlighting}
\end{Shaded}

\end{document}
