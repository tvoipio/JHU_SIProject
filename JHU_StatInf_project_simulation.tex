\documentclass[12pt,a4paper]{article}
\usepackage{lmodern}
\usepackage{amssymb,amsmath}
\usepackage{ifxetex,ifluatex}
\usepackage{fixltx2e} % provides \textsubscript
\ifnum 0\ifxetex 1\fi\ifluatex 1\fi=0 % if pdftex
  \usepackage[T1]{fontenc}
  \usepackage[utf8]{inputenc}
\else % if luatex or xelatex
  \ifxetex
    \usepackage{mathspec}
    \usepackage{xltxtra,xunicode}
  \else
    \usepackage{fontspec}
  \fi
  \defaultfontfeatures{Mapping=tex-text,Scale=MatchLowercase}
  \newcommand{\euro}{€}
    \setmainfont{texgyretermes-regular.otf}
    \setmathfont(Digits,Latin,Greek){texgyretermes-math.otf}
\fi
% use upquote if available, for straight quotes in verbatim environments
\IfFileExists{upquote.sty}{\usepackage{upquote}}{}
% use microtype if available
\IfFileExists{microtype.sty}{%
\usepackage{microtype}
\UseMicrotypeSet[protrusion]{basicmath} % disable protrusion for tt fonts
}{}
\usepackage[margin=1in]{geometry}
\ifxetex
  \usepackage[setpagesize=false, % page size defined by xetex
              unicode=false, % unicode breaks when used with xetex
              xetex]{hyperref}
\else
  \usepackage[unicode=true]{hyperref}
\fi
\hypersetup{breaklinks=true,
            bookmarks=true,
            pdfauthor={Timo Voipio},
            pdftitle={Simulated distribution of the mean of exponentially distributed random variables},
            colorlinks=true,
            citecolor=blue,
            urlcolor=blue,
            linkcolor=magenta,
            pdfborder={0 0 0}}
\urlstyle{same}  % don't use monospace font for urls
\usepackage{color}
\usepackage{fancyvrb}
\newcommand{\VerbBar}{|}
\newcommand{\VERB}{\Verb[commandchars=\\\{\}]}
\DefineVerbatimEnvironment{Highlighting}{Verbatim}{commandchars=\\\{\}}
% Add ',fontsize=\small' for more characters per line
\usepackage{framed}
\definecolor{shadecolor}{RGB}{248,248,248}
\newenvironment{Shaded}{\begin{snugshade}}{\end{snugshade}}
\newcommand{\KeywordTok}[1]{\textcolor[rgb]{0.13,0.29,0.53}{\textbf{{#1}}}}
\newcommand{\DataTypeTok}[1]{\textcolor[rgb]{0.13,0.29,0.53}{{#1}}}
\newcommand{\DecValTok}[1]{\textcolor[rgb]{0.00,0.00,0.81}{{#1}}}
\newcommand{\BaseNTok}[1]{\textcolor[rgb]{0.00,0.00,0.81}{{#1}}}
\newcommand{\FloatTok}[1]{\textcolor[rgb]{0.00,0.00,0.81}{{#1}}}
\newcommand{\ConstantTok}[1]{\textcolor[rgb]{0.00,0.00,0.00}{{#1}}}
\newcommand{\CharTok}[1]{\textcolor[rgb]{0.31,0.60,0.02}{{#1}}}
\newcommand{\SpecialCharTok}[1]{\textcolor[rgb]{0.00,0.00,0.00}{{#1}}}
\newcommand{\StringTok}[1]{\textcolor[rgb]{0.31,0.60,0.02}{{#1}}}
\newcommand{\VerbatimStringTok}[1]{\textcolor[rgb]{0.31,0.60,0.02}{{#1}}}
\newcommand{\SpecialStringTok}[1]{\textcolor[rgb]{0.31,0.60,0.02}{{#1}}}
\newcommand{\ImportTok}[1]{{#1}}
\newcommand{\CommentTok}[1]{\textcolor[rgb]{0.56,0.35,0.01}{\textit{{#1}}}}
\newcommand{\DocumentationTok}[1]{\textcolor[rgb]{0.56,0.35,0.01}{\textbf{\textit{{#1}}}}}
\newcommand{\AnnotationTok}[1]{\textcolor[rgb]{0.56,0.35,0.01}{\textbf{\textit{{#1}}}}}
\newcommand{\CommentVarTok}[1]{\textcolor[rgb]{0.56,0.35,0.01}{\textbf{\textit{{#1}}}}}
\newcommand{\OtherTok}[1]{\textcolor[rgb]{0.56,0.35,0.01}{{#1}}}
\newcommand{\FunctionTok}[1]{\textcolor[rgb]{0.00,0.00,0.00}{{#1}}}
\newcommand{\VariableTok}[1]{\textcolor[rgb]{0.00,0.00,0.00}{{#1}}}
\newcommand{\ControlFlowTok}[1]{\textcolor[rgb]{0.13,0.29,0.53}{\textbf{{#1}}}}
\newcommand{\OperatorTok}[1]{\textcolor[rgb]{0.81,0.36,0.00}{\textbf{{#1}}}}
\newcommand{\BuiltInTok}[1]{{#1}}
\newcommand{\ExtensionTok}[1]{{#1}}
\newcommand{\PreprocessorTok}[1]{\textcolor[rgb]{0.56,0.35,0.01}{\textit{{#1}}}}
\newcommand{\AttributeTok}[1]{\textcolor[rgb]{0.77,0.63,0.00}{{#1}}}
\newcommand{\RegionMarkerTok}[1]{{#1}}
\newcommand{\InformationTok}[1]{\textcolor[rgb]{0.56,0.35,0.01}{\textbf{\textit{{#1}}}}}
\newcommand{\WarningTok}[1]{\textcolor[rgb]{0.56,0.35,0.01}{\textbf{\textit{{#1}}}}}
\newcommand{\AlertTok}[1]{\textcolor[rgb]{0.94,0.16,0.16}{{#1}}}
\newcommand{\ErrorTok}[1]{\textcolor[rgb]{0.64,0.00,0.00}{\textbf{{#1}}}}
\newcommand{\NormalTok}[1]{{#1}}
\usepackage{graphicx,grffile}
\makeatletter
\def\maxwidth{\ifdim\Gin@nat@width>\linewidth\linewidth\else\Gin@nat@width\fi}
\def\maxheight{\ifdim\Gin@nat@height>\textheight\textheight\else\Gin@nat@height\fi}
\makeatother
% Scale images if necessary, so that they will not overflow the page
% margins by default, and it is still possible to overwrite the defaults
% using explicit options in \includegraphics[width, height, ...]{}
\setkeys{Gin}{width=\maxwidth,height=\maxheight,keepaspectratio}
\setlength{\parindent}{0pt}
\setlength{\parskip}{6pt plus 2pt minus 1pt}
\setlength{\emergencystretch}{3em}  % prevent overfull lines
\providecommand{\tightlist}{%
  \setlength{\itemsep}{0pt}\setlength{\parskip}{0pt}}
\setcounter{secnumdepth}{0}

%%% Use protect on footnotes to avoid problems with footnotes in titles
\let\rmarkdownfootnote\footnote%
\def\footnote{\protect\rmarkdownfootnote}

%%% Change title format to be more compact
\usepackage{titling}

% Create subtitle command for use in maketitle
\newcommand{\subtitle}[1]{
  \posttitle{
    \begin{center}\large#1\end{center}
    }
}

\setlength{\droptitle}{-2em}
  \title{Simulated distribution of the mean of exponentially distributed random
variables}
  \pretitle{\vspace{\droptitle}\centering\huge}
  \posttitle{\par}
  \author{Timo Voipio}
  \preauthor{\centering\large\emph}
  \postauthor{\par}
  \predate{\centering\large\emph}
  \postdate{\par}
  \date{11 Aug 2016}


% Redefines (sub)paragraphs to behave more like sections
\ifx\paragraph\undefined\else
\let\oldparagraph\paragraph
\renewcommand{\paragraph}[1]{\oldparagraph{#1}\mbox{}}
\fi
\ifx\subparagraph\undefined\else
\let\oldsubparagraph\subparagraph
\renewcommand{\subparagraph}[1]{\oldsubparagraph{#1}\mbox{}}
\fi

\begin{document}
\maketitle

\section{Overview}\label{overview}

This document explores the distribution of the mean of 40 exponentially
distributed independent random variables via computer simulation. The
empirically determined distribution is found to be in good agreement
with the central limit theorem.

\section{Simulation of the mean}\label{simulation-of-the-mean}

We investigate the distribution of the random variable
\(\bar{X} = \sum_{i=1}^{40} Y_i\), where \(Y_i\) are \emph{iid},
exponentially distributed random variables with the parameter
\(\lambda = 0.2\). The simulation is carried out by drawing \(40000\)
numbers from an exponential distribution with \(\text{rate} = \lambda\),
grouping these into 1000 groups of 40 each, and calculating the
arithmetic mean for each group.

\section{Simulation results}\label{simulation-results}

The distribution of the averages obtained from the simulation is shown
in the following figure.

\begin{center}\includegraphics{JHU_StatInf_project_simulation_files/figure-latex/distfig-1} \end{center}

The vertical bars show the (density) histogram of the distribution of
\(\bar{X}\), and the black curve shows the density. The black vertical
line indicates the mean of \(\bar{X}\), i.e., the empirical mean. The
red dashed curve shows the (theoretical) distribution of \(\bar{X}\),
while the dashed vertical line presents the population mean of
\(\bar{X}\).

\subsection{Mean of averages}\label{mean-of-averages}

The sample mean of \(\bar{X}\) is found to be 5.033. Due to the
linearity of the expectation value, the expected value of \(\bar{X}\) is
equal to the expected value of an exponentially distributed random
variable with rate parameter \(\lambda\), i.e.
\(1/\lambda = 1/0.2 = 5\). The empirical mean deviates less than 1 \%
from the theoretical prediction.

\subsection{Variance and distribution of
averages}\label{variance-and-distribution-of-averages}

The sample variance of \(\bar{X}\) is 0.654. The central limit theorem
(CLT) predicts that, for sufficiently large \(n\), the average of \(n\)
\emph{iid} random variables has the variance \(\sigma^2/n\), where
\(\sigma^2\) is the variance of the distrubution which the random
variables follow. In the case of the exponential distribution,
\(\sigma^2 = 1/\lambda^2\), and therefore the CLT predicts that the
variance of \(\bar{X}\) is
\(1/\lambda^2 n = 1/(0.2^2 \times 40) = 0.625\). The empirical result
agrees very well with the prediction given by CLT, so it seems that, at
least in this case, \(n = 40\) is large enough for the central limit
theorem to be applicable.

\subsection{Distribution of sample
means}\label{distribution-of-sample-means}

The central limit theorem predicts that the mean of a large number of
\emph{iid} random variables follows the normal distribution. In the
figure above, the dashed red curve indicates the probability density of
random variables following the distribution \(N(5, 25/40)\). This
distrubution is the one predicted by CLT for the mean of \(n = 40\)
\emph{iid} random variables with the mean of \(1/\lambda = 5\) and
variance of \(1/\lambda^2 = 25\). Compared to the empirically determined
probability density, we see that the theoretical prediction works quite
well. The number of simulations used is not very large, so there is
bound to be some Monte Carlo ``air'' in the result. The appendix
presents and discusses the results of a comparable simulation conducted
with a significantly larger sample size.

\section{Conclusions}\label{conclusions}

We determined empirically the distribution of the average of 40
\emph{iid}, exponentially distributed random variables. The resulting
distribution and its mean and variance was compared to the normal
distribution predicted by the central limit theorem (CLT). The empirical
results agreed very well with the prediction given by CLT, and we
conclude that, at least in this particular case, 40 may be considered to
be a ``large'' number.

\section{Appendix}\label{appendix}

This appendix includes the R code used to conduct the simulation
experiment and to format the results. The simulation was performed using
R version 3.2.3 (2015-12-10) on OS X 10.10.5 (Yosemite)
{[}x86\_64-apple-darwin13.4.0 (64-bit){]}.

\subsection{Source code}\label{source-code}

\begin{Shaded}
\begin{Highlighting}[]
\KeywordTok{library}\NormalTok{(ggplot2)}

\CommentTok{# Set random seed for repeatability}
\KeywordTok{set.seed}\NormalTok{(}\DecValTok{160808}\NormalTok{)}
\NormalTok{Nmean <-}\StringTok{ }\DecValTok{40}
\NormalTok{Nsim <-}\StringTok{ }\DecValTok{1000}
\NormalTok{rate <-}\StringTok{ }\FloatTok{0.2}

\CommentTok{# Generate the requisite amount of random numbers from the exponential}
\CommentTok{# distribution, wrape them into a Nsim x Nmean matrix, calculate}
\CommentTok{# row means}
\NormalTok{expmeans <-}\StringTok{ }\KeywordTok{apply}\NormalTok{(}\KeywordTok{matrix}\NormalTok{(}\KeywordTok{rexp}\NormalTok{(Nmean*Nsim, }\DataTypeTok{rate=}\NormalTok{rate), }\DataTypeTok{nrow=}\NormalTok{Nsim),}
                  \DecValTok{1}\NormalTok{, mean)}
\CommentTok{# Sample mean of Xbar}
\NormalTok{samplemeanavg <-}\StringTok{ }\KeywordTok{mean}\NormalTok{(expmeans)}
\CommentTok{# Population mean of Xbar}
\NormalTok{popmeanavg <-}\StringTok{ }\DecValTok{1}\NormalTok{/rate}

\CommentTok{# Sample variance of Xbar}
\NormalTok{samplevaravg <-}\StringTok{ }\KeywordTok{var}\NormalTok{(expmeans)}
\CommentTok{# Population variance of Xbar}
\NormalTok{popvaravg <-}\StringTok{ }\NormalTok{(}\DecValTok{1}\NormalTok{/rate^}\DecValTok{2}\NormalTok{)/Nmean}

\CommentTok{# A function for constructing a histogram+density plot of the}
\CommentTok{# simulated averages, along with the CLT prediction.}
\CommentTok{# (Placing the code inside a function enables reuse in the Appendix)}
\NormalTok{meandistplot <-}\StringTok{ }\NormalTok{function(means, popmean, popvar, samplemean)}
\NormalTok{\{}
    \CommentTok{# Calculate suitable bin widths}
    \NormalTok{binwidth <-}\StringTok{ }\KeywordTok{ceiling}\NormalTok{(}\KeywordTok{sqrt}\NormalTok{(popvar)*}\DecValTok{2}\NormalTok{)/}\DecValTok{10}
    \CommentTok{# Shift the bin origin such that integers values coincide with}
    \CommentTok{# bin centers}
    \NormalTok{binorigin <-}\StringTok{ }\NormalTok{binwidth*}\KeywordTok{floor}\NormalTok{(}\KeywordTok{min}\NormalTok{(means)/binwidth)-binwidth/}\DecValTok{2}
    \CommentTok{# Calculate the limits of the x axis such that at least 3}
    \CommentTok{# standard deviations (of the CLT normal distribution)}
    \CommentTok{# are included; mean predicted by the CLT is at the center of}
    \CommentTok{# the x axis}
    \NormalTok{xlims <-}\StringTok{ }\KeywordTok{ceiling}\NormalTok{(}\DecValTok{3}\NormalTok{*}\KeywordTok{sqrt}\NormalTok{(popvar))*}\KeywordTok{c}\NormalTok{(-}\DecValTok{1}\NormalTok{, }\DecValTok{1}\NormalTok{) +}\StringTok{ }\NormalTok{popmean}
    
    \CommentTok{# Construct the plot using ggplot2}
    \NormalTok{g <-}\StringTok{ }\KeywordTok{ggplot}\NormalTok{(}\KeywordTok{data.frame}\NormalTok{(}\DataTypeTok{expmean =} \NormalTok{means), }\KeywordTok{aes}\NormalTok{(}\DataTypeTok{x =} \NormalTok{expmean))}
    
    \CommentTok{# Histogram and density of the simulated data}
    \NormalTok{ghist <-}\StringTok{ }\NormalTok{g +}\StringTok{ }\KeywordTok{geom_histogram}\NormalTok{(}\KeywordTok{aes}\NormalTok{(}\DataTypeTok{y =} \NormalTok{..density..),}
                                \DataTypeTok{binwidth =} \NormalTok{binwidth, }\DataTypeTok{origin =} \NormalTok{binorigin) +}
\StringTok{        }\KeywordTok{geom_density}\NormalTok{() +}
\StringTok{        }\KeywordTok{geom_vline}\NormalTok{(}\DataTypeTok{xintercept =} \KeywordTok{mean}\NormalTok{(means), }\DataTypeTok{color =} \StringTok{"black"}\NormalTok{) +}
\StringTok{        }\KeywordTok{labs}\NormalTok{(}\DataTypeTok{x =} \KeywordTok{expression}\NormalTok{(}\KeywordTok{bar}\NormalTok{(X)),}
             \DataTypeTok{title =} \StringTok{"Distribution of simulated means"}\NormalTok{) +}
\StringTok{        }\KeywordTok{scale_x_continuous}\NormalTok{(}\DataTypeTok{breaks =} \KeywordTok{seq}\NormalTok{(xlims[}\DecValTok{1}\NormalTok{], xlims[}\DecValTok{2}\NormalTok{]),}
                           \DataTypeTok{limits =} \NormalTok{xlims) +}
\StringTok{        }\KeywordTok{theme}\NormalTok{(}\DataTypeTok{text =} \KeywordTok{element_text}\NormalTok{(}\DataTypeTok{family =} \StringTok{"serif"}\NormalTok{))}
    
    \CommentTok{# Determine the shape of the distribution of Xbar predicted by}
    \CommentTok{# CLT}
    \NormalTok{densx <-}\StringTok{ }\KeywordTok{seq}\NormalTok{(xlims[}\DecValTok{1}\NormalTok{], xlims[}\DecValTok{2}\NormalTok{], }\DataTypeTok{length.out =} \DecValTok{101}\NormalTok{)}
    \NormalTok{densy <-}\StringTok{ }\KeywordTok{dnorm}\NormalTok{(densx, }\DataTypeTok{mean =} \NormalTok{popmean, }\DataTypeTok{sd =} \KeywordTok{sqrt}\NormalTok{(popvar))}
    \NormalTok{densdata <-}\StringTok{ }\KeywordTok{data.frame}\NormalTok{(}\DataTypeTok{densx =} \NormalTok{densx, }\DataTypeTok{densy =} \NormalTok{densy)}
    
    \CommentTok{# Add the mean and density predicted by CLT to the plot}
    \NormalTok{gcomp <-}\StringTok{ }\NormalTok{ghist +}\StringTok{ }\KeywordTok{geom_line}\NormalTok{(}\KeywordTok{aes}\NormalTok{(}\DataTypeTok{x =} \NormalTok{densx, }\DataTypeTok{y =} \NormalTok{densy), }\DataTypeTok{data =} \NormalTok{densdata,}
                               \DataTypeTok{color =} \StringTok{"red"}\NormalTok{, }\DataTypeTok{lty =} \StringTok{"dashed"}\NormalTok{) +}
\StringTok{        }\KeywordTok{geom_vline}\NormalTok{(}\DataTypeTok{xintercept =} \NormalTok{popmean, }\DataTypeTok{color =} \StringTok{"red"}\NormalTok{, }\DataTypeTok{lty =} \StringTok{"dashed"}\NormalTok{)}
    
    \KeywordTok{return}\NormalTok{(gcomp)}
\NormalTok{\}}

\CommentTok{# tidyprint suppresses warnings issued by ggplot2 in the case a bin}
\CommentTok{# does not contain any values; this may happen since the limits have}
\CommentTok{# been set manually}
\CommentTok{# (further simulation with Nsim = 1e6 predicts that, for lambda used}
\CommentTok{# here, the 1/1000th quantile is at approximately 2.9; for Nsim = 1000,}
\CommentTok{# it is thus rather improbable that all of the bins 2.1-2.3, 2.3-2.5,}
\CommentTok{# 2.5-2.7, 2.5-2.9 would contain at least one observation)}
\NormalTok{tidyprint <-}\StringTok{ }\NormalTok{function(obj, tidy)}
\NormalTok{\{}
    \CommentTok{# If tidy output is desired, disable warnings (caused by xlim}
    \CommentTok{# containing bins with no values falling in them)}
    \NormalTok{if (tidy)}
    \NormalTok{\{}
        \NormalTok{oldw <-}\StringTok{ }\KeywordTok{getOption}\NormalTok{(}\StringTok{"warn"}\NormalTok{)}
        \KeywordTok{options}\NormalTok{(}\DataTypeTok{warn =} \NormalTok{-}\DecValTok{1}\NormalTok{)}
    \NormalTok{\}}
    
    \CommentTok{# Print the object using the default print method}
    \KeywordTok{print}\NormalTok{(obj)}
    
    \CommentTok{# Restore previous state of warnings}
    \NormalTok{if(tidy) \{ }\KeywordTok{options}\NormalTok{(}\DataTypeTok{warn =} \NormalTok{oldw) \}}
\NormalTok{\}}

\CommentTok{# Construct and print a plot of the simulated averages along}
\CommentTok{# with the CLT prediction}
\KeywordTok{tidyprint}\NormalTok{(}\KeywordTok{meandistplot}\NormalTok{(expmeans, popmeanavg, popvaravg, samplemeanavg),}
          \NormalTok{tidydoc)}
\end{Highlighting}
\end{Shaded}

The complete R Markdown source file may be found at
\url{https://github.com/tvoipio/JHU_SIProject}

\subsection{\texorpdfstring{Simulation using
\(n = 1000000\)}{Simulation using n = 1000000}}\label{simulation-using-n-1000000}

Out of interest (and because computing is cheap), it was decided to
carry out the simulation of the average of 40 \emph{iid} exponentially
distributed random variables using a million simulations, instead of one
thousand.

\begin{Shaded}
\begin{Highlighting}[]
\NormalTok{Nsim2 <-}\StringTok{ }\FloatTok{1e6}
\NormalTok{expmeans2 <-}\StringTok{ }\KeywordTok{apply}\NormalTok{(}\KeywordTok{matrix}\NormalTok{(}\KeywordTok{rexp}\NormalTok{(Nmean*Nsim2, }\DataTypeTok{rate=}\NormalTok{rate), }\DataTypeTok{nrow=}\NormalTok{Nsim2),}
                   \DecValTok{1}\NormalTok{, mean)}
\end{Highlighting}
\end{Shaded}

\begin{Shaded}
\begin{Highlighting}[]
\CommentTok{# Calculate sample mean and sample variance of the averages calculated}
\CommentTok{# using Nsim2 simulations and print the resulting distribution}
\NormalTok{samplemeanavg2 <-}\StringTok{ }\KeywordTok{mean}\NormalTok{(expmeans2)}
\NormalTok{samplevaravg2 <-}\StringTok{ }\KeywordTok{var}\NormalTok{(expmeans2)}
\KeywordTok{tidyprint}\NormalTok{(}\KeywordTok{meandistplot}\NormalTok{(expmeans2, popmeanavg, popvaravg, samplemeanavg2),}
          \NormalTok{tidydoc)}
\end{Highlighting}
\end{Shaded}

\begin{center}\includegraphics{JHU_StatInf_project_simulation_files/figure-latex/unnamed-chunk-4-1} \end{center}

The sample mean of the 1000000 simulated averages is 4.999539 (CLT
prediction 5.00000), and the sample variance is 0.6254257 (CLT
prediction 0.62500). We observe that, by increasing the number of
simulations the sample mean is, for most practical purposes, exactly the
same the as the prediction given by the linearity of the expected value.
The variance is also closer to the CLT prediction than the one obtained
with 1000 simulations (0.6541418), but not as close as in the case of
the mean. It should be kept in mind that the CLT is exact only as \(n\)
approaches infinity; in our case, \(n = 40\), so it is not surprising
that the predicted and empirically determined variances differ.

The shape of the distribution obtained with 1000000 simulations is also
clearly not normal; the distribution is positively skewed, the lower
tail is thinner than normal, and the upper tail is fatter than normal.
Determining the exact distribution, while possibly an interesting
exercise, is way, way outside the scope of this project.

\end{document}
